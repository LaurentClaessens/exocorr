% This is the example file for SystemeCorr.sty
% Remarks are put in comments.

\documentclass[a4paper,12pt]{book}

\usepackage{latexsym}
\usepackage{amsfonts}
\usepackage{amsmath}
\usepackage{amsthm}
\usepackage{amssymb}

\usepackage[utf8]{inputenc}
\usepackage[T1]{fontenc}

\usepackage{hyperref}			% Almost obligatory
\usepackage{ifthen}			% mandatory
\usepackage{SystemeCorr}		% Invoke the package
					% has to be put after inpuenc and fontenc because of the french ``à`` and ''é``

\usepackage{textcomp}
\usepackage{lmodern}
\usepackage[a4paper]{geometry} 
\usepackage[english,frenchb]{babel}


\begin{document}

\corrPolitique{2}	% Set to	0 for no corrections at all
			%		1 for having the corrections bellow each exercises (handy when working on the text)
			%		2 for having the corrections in a separated chapter (a link is then created)
			% See \corrChapitre bellow

% In order to create links to wikipedia, use
% \wikipedia{<language>}{<article>}{<text>}
% This will print <text> in your document and turn it into a link to
% http://<language>.wikipedia.org/wiki/<article>
% <language> can be fr, en, nl, etc.
Ceci est une lien vers wikipedia avec des accents : \wikipedia{fr}{Développement_de_Taylor}{Wikipédia} (en français, cela peut être un problème). This is a link to the \wikipedia{nl}{Nederlandstalige_Wikipedia}{Nederlandstalige} wikipedia.

\chapter{Exercises}

% The following lines are copy-pasted from the output of the command line
% liste_exo.py Example 1 3
\Exo{Example0001}
\Exo{Example0002}
\Exo{Example0003}

% If \corrPolitique{2}, the following line creates a chapter (I let you guess the title) which contains the corrections
% If \corrPolitique{0} or \corrPolitique{1}, does nothing
\corrChapitre{Correction of some exercises}


\end{document}
